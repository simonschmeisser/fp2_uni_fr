\section{Theoretical Basis}

\subsection{Atomar spectra, fine structure and hyperfine structure}

The nucleus of an atom induces an electric field, which binds electrons in orbits around it. Those orbits are quantized according to the boundary conditions the electron's wave function underlies in the Coulomb field. The electron's state is classically described by its principal quantum number $n$ and it's angular momentum $L$. The quantum numbers for all electrons define the atomic spectrum since the absorption and emission of photons depends on the electron's state (see ??).

If one considers the intrinsic angular momentum of the electron, the so called spin $S$, one receives a new splitting of the electron states. One can calculate this by adding the spin to the angular momentum of the electron and thus defining a new quantum number, the total angular momentum:

$$ \vec J = \vec L + \vec S $$

The total angular momentum induces a magnetic momentum of the electron, since the electron is a charged particle:

$$\vec \mu_J = \vec \mu_L + \vec \mu_S = - \frac{\mu_B}{\hbar}(g_S\vec S + g_L\vec L) = -\frac{\mu_B}{\hbar}g_J\vec J$$

$\mu_B = \frac{e\hbar}{2m_0} = 9.27\cdot10^{-24}Am^2$ is the Bohr magneton and $g_S \approx 2$, $g_L = 1$ and $g_J$ are the Landé-g-factors. This magnetic momentum then splits up the energy levels of the electrons using the Zeeman-effect (see ??). This is called the fine structure of the atom. The energy of a state changes by:

$$ E_{FS} = -\vec\mu_S\cdot\vec B_L $$

If one additionally considers the angular momentum of the nucleus $I$, which results from the nuclear spin of the protons and neutrons, one receives even a new quantum number, the hyperfine total angular momentum:

$$ \vec F = \vec J + \vec I $$

The nuclear spin itself induces a magnetic momentum 

$$\mu_I = \frac{g_I\mu_K}{\hbar}\vec I$$

and a splitting of the energy levels given by:

$$E_{HFS} = -\vec\mu_I\cdot\vec B_J$$

Which leads to the difference between two levels:

$$\Delta E_{HFS} = \frac{1}{2}\underbrace{\frac{g_I\mu_KB_J}{\sqrt{J(J+1)}}}_{=:A}[F(F+1) + J(J+1) + I(I+1)]$$

If the levels are adjacent to each other, this becomes:

\begin{equation} \Delta E_{HFS}(\Delta F = 1) = A\cdot (F+1) \end{equation} 

\subsubsection{Hyperfine structure of Rubidium}

Rubidium has two stable isotopes:

\begin{center}
\begin{tabular}[H]{c c c c c}
Isotope & Proportion &  S & L & J \\
 $^{85}Rb$ & 72.8\% & 1/2 & 0 & 1/2 \\
 $^{87}Rb$ & 27.2\% & 1/2 & 1 & 1/2 \\
\end{tabular}
\end{center}

























