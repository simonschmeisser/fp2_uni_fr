\clearpage
\section{Summary}

In the first part of the experiment, we determined the characteristics of the laser diode. We found out, how the frequency of the emitted light of the laser diode changes by changing its temperature or its current. The following values describe the sought relationships:

\begin{center}
\begin{tabular}[H]{l l}
Frequency-Current & $\Delta\nu = \Delta I \cdot (-3.858 \pm 0.056)\ {Ghz}/{mA}$\\
Frequency-Temperature & $\Delta\nu = \Delta T \cdot (-31.70 \pm 1.12)\ {Ghz}/{^\circ C}$\\
\end{tabular}\\
\end{center}

We then acquired the hyperfine structure spectrum of the two Rubidium isotopes. By fitting the spectrum with eight Gauss peaks, and the calibration with an etalon, we were able to find the relative frequency of the peaks to each other and herewith determine the interval constants of the hyperfine structure.

\begin{center}
\begin{tabular}[H]{| c | c c c c |} \hline
Isotope & $A_1$ in $\mu$eV & $A_2$ in $\mu$eV & $A_{th.}$ in $\mu$eV & St. dev.\\ \hline
$^{87}Rb$, $^2S_{1/2}$ & $16.19 \pm 0.51$ & $16.20 \pm 0.51$ & 14.13  & 5 \\
$^{87}Rb$, $^2P_{1/2}$ & $1.79 \pm 0.51$ &  $1.77 \pm 0.50$ & 1.692   & 1 \\
$^{85}Rb$, $^2S_{1/2}$ & $5.63 \pm 0.33$ &  $5.37 \pm 0.34$ & 4.185   & 4 - 5 \\
$^{85}Rb$, $^2P_{1/2}$ & $0.37 \pm 0.34$ &  $0.62 \pm 0.33$ & 0.499   & 1 \\ \hline
\end{tabular}\\
\end{center}

In the third part, we used the double resonance method to calculate the earth's horizontal and vertical magnetic field. The experimental setup had to be turned into north-south direction in order to ensure measurements of high quality. The following values were determined:

\begin{center}
\begin{tabular}[H]{| l | c c c c |} \hline
 & $B_1 /\mu T$ & $B_2 /\mu T$ & $B_{theo} /\mu T$ & St. dev.\\ \hline
horizontal field & $12.38 \pm 1.13$ & $12.38 \pm 1.13$ &  20.9 & 8 \\
vertical field & $38.56 \pm 2.38$ & $39.51 \pm 2.38$ & 42.9 & 2 \\ \hline
\end{tabular}
\end{center}

Also, with this method we were able to determine the nuclear spin of the two Rubidium isotopes:

\begin{center}
\begin{tabular}[H]{| c | c c c c |} \hline
Isotope & Nuclear Spin & Theory & Error & St. dev. \\ \hline
 $^{87}Rb$ & $1.39 \pm 0.02$ & 1.5 & 7.3\% & 6 \\
 $^{85}Rb$ & $2.42 \pm 0.01$ & 2.5 & 3.2\% & 8\\  \hline
\end{tabular}
\end{center}

In part 4, we made the Rubidium atoms precess around the earth's vertical magnetic field, and from the precession frequency, we could again determine the earth's vertical magnetic field:

$$B_{vert} = (40.60\pm0.17) \mu T $$

which lies in the 14. standard deviation of the theoretical field.

Also did we change the vertical field and calculated the precession frequency in dependence of it and found, that 

$$\nu_{osz} = bx + a = (4.78\pm0.04)B_{vert} + (187.52\pm1.13)$$

Finally, we tried to measure the relaxation time using Dehmelt's and Franzen's method and found:

\begin{center}
\begin{tabular}[H]{c c}
		& $T_R$/ms \\
Dehmelt 	& $5.59 \pm 5.17$ \\
Franzen 	& $0.69 \pm 0.28$ \\
\end{tabular}
\end{center}

We found out that the value from Franzen's method is much more significant because the value from Dehmelt's method has an error as large as the value itself. We calculated the weighted average and found

$$T_R = (0.70 \pm 0.28) ms$$







