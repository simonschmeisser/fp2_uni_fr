\documentclass[a4paper,oneside,bibtotocnumbered]{scrartcl} %twocolumn,
\usepackage[utf8]{inputenc} %für MAC: applemac; für Windows: latin1 statt utf8
\usepackage[T1]{fontenc}
\usepackage[ngerman]{babel}

\usepackage{amsmath}
\usepackage{amsfonts}
\usepackage{amssymb}

\usepackage{mathptmx}
\usepackage{microtype}
\usepackage[nice]{nicefrac}

\usepackage{booktabs}
\usepackage{graphicx}
\usepackage{wrapfig}
\usepackage{float}
\usepackage[colorlinks=true, unicode]{hyperref}

\setkomafont{captionlabel}{\upshape\bfseries}
\setkomafont{caption}{\itshape}


\title{Versuchsprotokoll FP 2\\
\rule{0.5\textwidth}{0.4pt}\\
\vspace*{1cm}
\begin{Huge}
Holographie 
\end{Huge}
}
\subtitle{\vspace*{1cm}
\includegraphics[width=0.7\textwidth]{Photos/IMG_3909.jpg}
\vspace*{1.5cm}}

\author{Robi Pedersen \and Simon Schmeißer}
\date{Versuchsdurchführung 07.04. - 11.04.2011
\vspace*{1cm}\\
\rule{0.8\textwidth}{0.4pt}\\
Physikalisches Institut, Albert-Ludwigs-Universität Freiburg
}

\hypersetup{
pdftitle={Holographie},
pdfauthor={Robi Pedersen, Simon Schmeißer}}

\providecommand{\e}[1]{\ensuremath{\times 10^{#1}}}

\begin{document}
\begin{titlepage}
  \maketitle
  \vfill
  \thispagestyle{empty}
\end{titlepage}

\tableofcontents
\clearpage
%*****************************************************************

\section{Aufgabenstellung}

\begin{enumerate}
\item Prüfen Sie mit Hilfe des Michelson-Interferometers die Empfindlichkeit der optischen Bank auf  äußere Einflüsse und untersuchen Sie die Kohärenzeigenschaften des Lasers.
\item Vermessen Sie die Durchbiegung von drei einseitig eingespannten Metallbalken aus Aluminium, Messing und Stahl bei Belastung mit einer sehr geringen Kraft anhand eines Doppelbelichtungshologramms.
\item Untersuchen Sie mittels Echtzeitholographie die Eigenschwingung einer eingespannten Aluminiumplatte.
\item Beobachten Sie mit Hilfe der Fourierspektroskopie die Kreuzkorrelationsfunktion zweier gegeneinander verdrehter Spalte.
\end{enumerate}
\section{Theorie}

\subsection{Bildung des Positroniums ($e^+ e^-$)}

Werden Positronen \positron in Materie gestoppt, so geben sie zunächst durch zahlreiche inelastische Stöße ihre Energie ab, bevor es mit Elektronen \elektron der äußeren Schalen zur Paarvernichtung unter Emission elektromagnetischer Strahlung kommt.
\begin{equation*}
 \positron + \elektron \rightarrow 2\gamma , 3\gamma
\end{equation*}

Dieser Prozess kann zur Abstrahlung von zwei oder drei Photonen führen, wobei der $e\gamma$-Zerfall ungefähr um den Faktor $\alpha = \frac{e^2}{\hbar c} \sim \frac{1}{137}$ unterdrückt ist. Der Wirkungsquerschnitt $\sigma_{2\gamma}$ eines ruhenden Elektrons, in nezug auf Vernichtung mit dem Positron in 2 Quanten, lautet für nichtrelativistische Geschwindigkeiten ($ v << c$) [LIT: DIRAC 30]:
\begin{equation*}
 \sigma_{2\gamma} = \frac{\pi r_0^2 c}{v}
\end{equation*}
mit
\begin{equation*}
 r_0 = \text{Thomson-Radius} = \frac{e^2}{mc^2} = 2,8\e{-13}\text{cm}
\end{equation*}

\begin{equation*}
 v = \text{Relativgeschwindigkeit von \positron und \elektron}
\end{equation*}

Dann ist die Vernichtungswahrscheinlichkeit eines Positrons in einem Medium, daß n Elektronen pro $\text{cm}^3$ enthält, gegeben durch:
\begin{equation*}
 \lambda = \frac{1}{\tau} = \sigma_{2\gamma} \cdot n \cdot v = \pi r_0^2 c n   \tau = \text{mittlere Lebensdauer}
\end{equation*}

woebi die Coulomb-Anziehung zwischen den Teilchen vernachlässigt wird.

Dieser direkten Vernichtung eines Positron-Elektron-Paares kann bei noedrigen Geschwindigkeiten die Entstehung des kurzlebigen Positronium-Atoms (\positron \elektron) vorrausgehen:

\begin{equation*}
 \positron + \elektron \rightarrow \positron \elektron \rightarrow 2\gamma, 3\gamma
\end{equation*}

Es sind nur Positronen eines schmalen Energiebereichs in der Lage, sich ein Elektron einzufangen und Positronium im Grundzustand zu bilden.
Da die Ionisierungsenergie von $6,8 \text{eV}$ des Positroniums kleiner als diejenige aller stabilen Gase ist, können Positronen deren $E_\text{kin} < (V_i - 6,8 \text{eV}$ ($V_i$ = Ionisationspotential der Gasmoleküle) ist, kein Positronium bilden. Andererseits sind aber bei großen kinetischen Energien die Wirkungsquerschnitte für Stoßionisation und Anregung der Gasmoleküle größer als bei den Einfangprozessen zur Bildung von Positronium. Eine Entstehung im 1. angeregten Zustand ist unter gewöhnlichen experimentellen Bedingungen sehr unwahrscheinlich, da die Bindungsenergie nur $1,7 \text{eV}$ beträgt. Das bedeutet, daß das Minimum der zum Einfangen eines Elektrons erforderlichen kinetischen Energie ($V_i - 1,7 \text{eV}$) bereits so groß ist, daß inelastische Stöße stark überwiegen. Wurde ien Positronium gebildet, do kann es natürlich in molekularen Stößen wieder in ein Positron und ein Elektron auseinanderbrechen. Da die mittlere freie Weglänge in Festkörpern kleiner ist als in verdünnten Medien, kann man in den amorphen Substanzen Teflon und geschmolzener Quarz nur ungefähr 1 \% Positronium nachweisen. Deshalb ist es vorteilhafter mit speziellen Gasen, wie z.B. Schwefelhexafluorid $\text{SF}_6$ oder Freon $\text{C Cl}_2\text{F}_2$, zu arbeiten, bei denen man einen Anteil von ungefähr 30 \% erhält.

\subsection{Vernichtungscharakteristik des Positroniums}

Der Grundzustand des Positroniums 1S (l=0) wird je nach Spineinstellung in 2 Arten unterschieden:
\begin{enumerate}
 \item[a)] - Spins von Elektronen \elektron und Positronen \positron parallel
Gesamtspin $S = S_\elektron + S_\positron = 1$
Bezeichnung: Triplett-Positronium oder Orthopositronium (${}^3S_1$, $m_S = 0$, $m_S = \pm 1$)
$m_S$ = magnetische Spinquantenzahl
Statistisches Gewicht: $\frac{3}{4}$

 \item[b)] - Spins von \elektron und \positron antiparallel
S= 0
Bezeichnung: Singulett-Positronium oder Parapositronium (${}^1S_0$, $m_S = 0$)
Statistisches Gewicht: $\frac{1}{4}$ 
\end{enumerate}

Um im Vernichtungsprozeß den Erhaltungssätzen zu genügen, unterscheiden sich Triplett- und Singulett-Zustände erheblich in ihrer Vernichtungscharakteristik
\begin{equation*}
 \positron + \elektron \rightarrow 2\gamma , 3\gamma
\end{equation*}
Wegen der verletzung der Energie- und Impulserhaltung ist ein $1\gamma$-Zerfall im freien Raum nicht möglich.
Der $1^1S$-Zustand hat eine mittlere Lebensdauer $\tau_{2\gamma}$ von:
\begin{equation*}
 \frac{1}{\tau_{2\gamma}} = \lambda = \sigma_{2\gamma} \cdot v \cdot \left| \Psi(0) \right|^2 = \frac{\alpha^5 m c^2}{2 \hbar}
\end{equation*}
\begin{equation*}
 \text{mit} \left| \Psi(0) \right|^2 = \frac{1}{\pi} \left( \frac{1}{2a_0} \right)^3
\end{equation*}
Dichte der 1S-Elektronen am Positron, Bohrscher Radius $a_0 = \frac{r_0}{\alpha^2}$
Numerisch: $\tau_{2\gamma} = 1,25\e{-10}\text{sec}$

Unter Berücksichtigung der Erhaltungssätze zerstrahlt er in zwei Photononen, die sich in entgegengesetzter Richtung mit je 0,511 MeV ausbreiten und entweder im gleichen Sinne zirkularpolarisiert oder zueinander orthogonal linearpolarisiert sind.

Für den ${}^3S$-Zustand verbieten mit der Symmetrie des Systems verknüpfte Auswahlregeln den Zerfall in zwei Photononen. Der wahrscheinlichste Zerfallsvorgang des Orthopositroniums ist ein Strahlungsprozeß dritter Ordnung. Es zerstrahlt in drei Quanten, die wegen der Impulserhaltung in einer Ebene emittiert werden und jeweils ein kontinuirliches Energiespektrum von 0 bis 2 $m_ec^2$ besitzen, so daß die Gesamtenergie $E_{3\gamma} = 1,022 MeV$ gleich der gesamten Ruheenergie der beiden einander vernichtenden Teilchen entspricht. Weiterhin ist die Dreiquantenvernichtung gekennzeichnet durch eine kleinere Zerfallsrate $\lambda$ als die Zweiquantenvernichtung. Für das Verhältnis dieser Raten sind mehrere Berechnungen gemacht worden [LIT:Ore 49]; es ergibt sich zu:
\begin{equation*}
 \frac{\lambda_{3\gamma}}{\lambda_{2\gamma}} = \frac{\tau_{2\gamma}}{\tau_{3\gamma}} = \frac{4}{9\pi} \left(\pi^2 - 9 \right) \alpha = \frac{1}{1115}
\end{equation*}

Die mittlere Lebensdauer der $1^3S$-Zustands ist:
\begin{equation*}
 \tau_{3\gamma} = 1,39\e{-7} \text{sec}
\end{equation*}

Die Entstehung der Vernichtungsauswahlregeln, die sich aus den Invariantenbedingungen der Operation Ladungskonjugation C und Parität P ergeben, wird im nächsten Abschnitt näher behandelt.

\subsubsection{Symmetrieauswahlregeln}
Zunächst soll gezeigt werden, daß der Singulett-Zustand des Positroniums noicht durch Emission von drei Photonen oder jeder anderen ungeraden Anzahl von Quanten vernichtet werden kann. Die Operation der Ladungskonjugation C transformiert das Teilchen in sein Antiteilchen und kehrt das Vorzeichen der elektromagnetischen Felder um.
Für das nach außenhin elektrisch neutrale System des Positroniums hat die Ladungskonjugation C einen definierten Eigenwert $C=\pm1$, der beim Positronium-Zerfall erhalten bleiben muß. Für das Photon gilt ungerade C-Parität $C=-1$ , und da C eine multiplikative Quantenzahl ist, besitzt das System der n Vernichtungsquanten den Eigenwert $(-1)^n$.

Die C-Parität des Anfangszustands gewinnt man aus der folgenden Überlegung. Es kann nämlich gezeigt werden, daß die Operation C, angewandt auf das Positron-Elektron-Paar, äquivalent dem Ladungsaustausch ist, der seinerseits Parität P x Spinaustausch $S_{ex}$ entspricht:

$C | \positron \elektron >$ = Teilchenaustausch =(TODO:entspricht) Ladungsaustausch = P x Spinaustausch $S_{ex}$

Zum Beweis dieser Äquivalenz sei folgender Zustand betrachtet (s.Skizze 1):
\begin{itemize}
 \item[-] \elektron befindet sich am Ort $\vec{r}$ mit Spin $\alpha$
 \item[-] \positron befindet sich am Ort $-\vec{r}$ mit Spin $\beta$
\end{itemize}

1. Operation: Anwendung des Paritätsoperators P
 (= TODO:entspricht Inversion aller Raumkoordinaten)

Ergebnis:
\begin{itemize}
 \item[-] \elektron befindet sich am Ort $-\vec{r}$ mit Spin $\alpha$
 \item[-] \positron befindet sich am Ort $\vec{r}$ mit Spin $\beta$
\end{itemize}
Da die Spins axiale Vektoren sind, werden sie bei räumlicher Spiegelung nicht verändert.

2. Operation: Spin-Austausch
Ergebnis:
\begin{itemize}
 \item[-] \elektron befindet sich am Ort $-\vec{r}$ mit Spin $\beta$
 \item[-] \positron befindet sich am Ort $\vec{r}$ mit Spin $\alpha$
\end{itemize}
Offensichtlich erhält man nach beiden Operationen einen Zustand, in dem lediglich die Ladungen gegenüber dem Anfangszustand vertauscht sind.

TODO: SEITE 9 EINFÜGEN

Jetzt werden die beiden Operationen, die der Ladungskonjugation C äquivalent sind, auf den Positronium-Zustand angewendet.

\begin{itemize}
 \item[a)] Paritätsoperator P:
Der Ortsanteil der Wellenfunktion, ausgedrückt in Kugelkoordinaten, hat $(-1)^l$-Symmetrie (l = Bahnquantenzahl).
Die Gesamtparität ist:
\begin{equation*}
 P = \eta_i( \positron ) \cdot \eta_i (\elektron) \cdot (-1)^l
\end{equation*}
$\eta_i$ = definierte Eigenparität der einzelnen Elementarteilchen
\begin{equation*}
 \eta_i ( \elektron ) = - \eta_i ( \positron )
\end{equation*}

Beim Positron im Grundzustand (l = 0) erhalten wir also bei Inversion der Raumkoordinaten den Faktor:
\begin{equation*}
 P = (-1) \cdot (+1) \cdot (+1)
\end{equation*}

\item[b)] Spinaustausch $S_{ex}$:
Beim Austausch der Spinkoordinaten von Elektron und Positron tritt in der Spinwellenfunktion ein Faktor $(-1)^{s+l}$ auf (S = Gesamtspin). Für den Singulett-Zustand (Spins antiparallel, S = 0) bedeutet das, daß die Spinwellenfunktion antisymmertrisch ist.
\begin{equation*}
 S_{ex} = (-1)
\end{equation*}
\end{itemize}

Insgesamt besitzt der Singulett-Zustand folgende C-Parität:
\begin{equation*}
 C = P \cdot S_{ex} \\
 C = (-1)(+1)(+1)\cdot(-1) \\
 C = (+1)
\end{equation*} %TODO: Mehrzeilige Gleichung

Da der Ladungskonjugations-Operator C eine Konstante der Bewegung ist $ \frac{dC}{dt} = [H, C] = 0$, kann der Singulett-Zustand also nur in zwei Quanten zerstrahlen; denn der C-Eigenwert für die Photonen beträgt $C = (-1)^n$. Ganz analog erfolgt die Berechnung für den Triplett-Zustand, aus welcher hervorgeht, daß er nur in eine ungerade zahl von Quanten zerfallen kann.


\section{Michelson-Interferometer}
\subsection{Versuchsbeschreibung}

Bei diesem Versuchsteil soll die Stabilität des Versuchsaufbaus, die Auswirkung äußerer Einflüsse sowie die Kohärenzlänge des verwendeten Lasers untersucht werden. Da uns ein neuer Versuchstisch zur Verfügung gestellt wurde, sollte auch ein Vergleich mit dem bisherigen durchgeführt werden.


\begin{figure}[ht]
 \includegraphics[width=\textwidth]{BilderAufbau/Michelson.pdf}
 \caption{Draufsicht auf das Interferometer}
 \label{aufbau_interferometer}
\end{figure}

Der Aufbau (Abb. \ref{aufbau_interferometer}) besteht aus dem Laser (a) und einem Michelson-Interferometer. Bei diesem wird am Strahlteiler (b) der Laserstrahl in zwei Komponenten geteilt. Ein Teil geht in Einfallrichtung gerade weiter und trifft auf den Spiegel (c), wo er reflektiert wird. Nachdem er zusätzlich am Strahlteiler reflektiert wurde, trifft er auf den Schirm. Der andere Teil wird bereits beim ersten Auftreffen auf den Strahlteiler reflektiert und gelangt über einen weitere Spiegel (d) ebenfalls auf den Schirm (f). Das dort beobachtete Interferenzmuster kann zusätzlich mit einem Mikroskopobjektiv (e) vergrößert werden.  

Ist der Versuchsaufbau stabil aufgebaut und sind die äußeren Störungen nicht zu groß, so sollten die Weglängen der beiden Strahlen auf dem Schirm nur orts- nicht jedoch zeitabhängig sein. Die zeitliche Stabilität des Interferenzmusters dient uns somit als Mass für die Qualität und Empfindlichkeit des Aufbaus.

Ist ein stabiles Interferenzmuster eingestellt, so kann die Kohärenzlänge des Lasers bestimmt werden. Dazu wird einer der beiden Spiegel (bei uns c) so lange verschoben bis das Interferenzmuster zusammenbricht. Der Weglängenunterschied zwischen b-d-b und b-c-b bei dem gerade noch Interferenz erkennbar ist, entspricht der Kohärenzlänge.


\subsection{Durchführung und Auswertung}

Wir haben das Michelson-Interferometer zuerst mit dem alten Tisch wie in Abb \ref{aufbau_interferometer} aufgebaut. Dabe
\begin{figure}[ht]
 \includegraphics[width=\textwidth]{Photos/IMG_3881.jpg}
 \caption{Interferenzmuster}
\end{figure}


\begin{figure}[ht]
 \includegraphics[width=\textwidth]{Photos/IMG_3887.jpg}
 \caption{Interferenzmuster}
\end{figure}
\section{Doppelbelichtung}
\subsection{Versuchsbeschreibung}

Bei der Doppelbelichtungstechnik wird ein Hologramm des zu untersuchenden Objekts aufgenommen. Dabei befindet sich dieses während einer Hälfte der Belichtungszeit im ersten Zustand und wird dann für den Rest der Zeit in den zweiten Zustand gebracht. Man sieht dann im fertigen Hologramm die Interferenz beider Zustände.

In unserem Fall soll das Biegeverhalten von Metallstäben untersucht werden. Diese werden dazu sowohl mit als auch ohne einem ziehenden Gewicht aufgenommen. Aus dem Interferenzmuster können dann die Weglängenunterschiede und somit die Biegung der Stäbe ermittlelt werden. 

\subsection{Durchführung und Auswertung}

Wie beim vorherigen Versuchsteil begründet, verwenden wir den neuen Tisch und haben den in Abb \ref{doppelbelichtung-aufbau} skizzierten Aufbau verwendet. Da der Laser hier deutlich niedriger positioniert war, haben wir ihn zuerst mit zwei Spiegeln (a, b) um einige Zentimeter erhöht. Dann wird der Strahl wieder mit dem Strahlteiler (c) aufgetrennt in einen Objekt- und einen Referenzstrahl. Der Referenzstrahl wird mit einem Spiegel (d) umgelenkt und trifft direkt auf den Film (i). Der Objektstrahl wird über einen weiteren Spiegel (f) umgelenkt, so dass er auf das Objekt (h) fällt. Seine Reflektion am Objekt interferiert dann mit dem Referenzstrahl auf dem Film und liefert das Hologrammbild. Soweit der einfache Teil. Um ein ausreichend großes und homogenes Leuchtfeld zu erreichen, wird der Strahl mit einem Mikroskopobjektiv aufgeweitet und ein Lochfilter in dessen Brennpunkt gebracht. Das Loch, das sog. "`pinhole"', mit einem Durchmesser von ca. $10\mu m$ muss dabei sehr genau justiert werden, was uns große Freude bereitete. Diese Konstruktion soll auch höhere Beugungsordnungen herausfiltern und wird daher Raumfilter genannt. Wir haben sie (e, g) zwischen den letzten Spiegeln und dem Objekt/Film eingebaut. Das Verhältnis von Objekt- zu Referenzstrahl auf dem Film sollte etwa 1:10 bis 1:30 betragen und mit der Photodiode justiert werden. Es stellte sich jedoch heraus, dass diese bei weitem nicht sensitiv genug ist. Wir haben aber den Strahlteiler sowieso so eingestellt das nahezu alle Intensität im Objektstrahl lag, da bei der Reflektion am Objekt große Streuverluste auftreten. 

Wir haben dann ein Bild mit den empfohlenen Einstellungen \cite{versuchsanleitung} aufgenommen und entwickelt, konnten aber im fertigen Bild nichts erkennen. Die angegebene Belichtungszeit von 50s scheint uns die Hauptursache dafür zu sein. Für die weiteren Aufnahmen haben wir dann die folgenden mündlich überlieferten Prozessschritte \cite{lena_christian} verwendet:
\begin{center}
\begin{table}[H]
\centering
\begin{tabular}{l}
 \toprule
 2 x 5min belichten\\
 2min entwickeln\\
 10s vorwässern\\
 2min wässern\\
 2min bleichen\\
 10s vorwässern\\
 10min wässern (laufend)\\
 1min wässern (Spülmittel) \\
 \bottomrule 
\end{tabular}
\caption{Prozessschritte für die Erstellung von Holographien auf Planfilmstücken}
\end{table}
\end{center}

Mit diesen erhielten wir dann ein gutes Hologramm (Abb. \ref{doppelbelichtung-links}), wobei allerdings der Aluminiumstab kaum beleuchtet war. Für diesen haben wir daher ein weiteres Bild (Abb. \ref{doppelbelichtung-rechts}) mit angepasster Beleuchtung aufgenommen. Unser Maskotchen Heinrich, den wir ebenfalls mit abbilden wollten, kann man leider nicht erkennen. 

\begin{figure}[H]
 \includegraphics[height=0.5\textheight]{BilderAufbau/doppelbelichtung.pdf}
 \caption{Schematischer Versuchsaufbau für die Doppelbelichtungsholographie}
 \label{doppelbelichtung-aufbau}
\end{figure}


\begin{figure}[H]
 \includegraphics[width=\textwidth]{Photos/IMG_3909.jpg}
 \caption{Hologramm der linken beiden Stäbe}
 \label{doppelbelichtung-links}
\end{figure}


\begin{figure}[H]
 \includegraphics[width=\textwidth]{Photos/IMG_3919.jpg}
 \caption{Hologramm des rechten Stabes}
 \label{doppelbelichtung-rechts}
\end{figure}

\begin{figure}[H]
 \includegraphics[width=\textwidth]{Photos/IMG_3919-korrigiert.jpg}
 \caption{Hologramm des rechten Stabes mit Perspektivenkorrektur}
 \label{doppelbelichtung-rechts-korrigiert}
\end{figure}
Das erste Minimum des Interferenzmusters ausgehend von der Befestigung entspricht einer Differenz von $\Delta x_1 = \nicefrac{\lambda}{2}$, die weiteren Minimas liegen dann bei $\Delta x_i = \Delta x_{i-1} + \lambda$. Wir haben also (falls nötig) eine Perspektivenkorrektur (Abb. \ref{doppelbelichtung-rechts-korrigiert}) vorgenommen und mit dem Programm {\verb Engauge Digitizer} über die Skala des Schirms die Position der Minima abgelesen und die Verbiegung berechnet. 

\begin{figure}[H]
 \includegraphics[width=\textwidth]{Graphen/biegung-balken1.pdf}
 \caption{Durchbiegung des Stahl-Stabes}
\end{figure}

\begin{figure}[H]
 \includegraphics[width=\textwidth]{Graphen/biegung-balken2.pdf}
 \caption{Durchbiegung des Messing-Stabes}
\end{figure}

\begin{figure}[H]
 \includegraphics[width=\textwidth]{Graphen/biegung-balken3.pdf}
 \caption{Durchbiegung des Aluminium-Stabes}
\end{figure}

In den Graphen haben wir dann die Verbiegung $y = (n + \nicefrac{1}{2}) \cdot \lambda$ mit n der Nummer des Minimums gegen deren Position aufgetrage. Wir erwarten hier also einen konvexen Funktionsverlauf. Dieser ist aus der Theorie durch 
$f(x) = p0 \cdot (5 x^2  - \frac{x^3}{6})$ gegeben. Wir ermitteln also durch Fitten den experimentellen Faktor $p0$ aus dem wir dann das Elastizitätsmodul bestimmen können: 
\[
 E = 12 \cdot \frac{F}{  p0 \cdot b \cdot c^3}
\]
wobei F die Gewichtskraft des 30g-Gewichts ist, b die Breite des Balkens und c seine Dicke (in Richtung der angreifenden Kraft). Der Fehler auf das Elastizitätsmodul setzt sich also zusammen aus dem Fehler auf den Fitparameter $p0$ sowie den Fehlern auf die Masse m, Breite b und Dicke c:

\[
  \sigma_E = E \cdot \sqrt{\left(\frac{\sigma_b}{b}\right)^2 + \left(\frac{\sigma_c}{c}\right)^2 + \left(\frac{\sigma_p0}{p0}\right)^2 + \left(\frac{\sigma_m}{m}\right)^2}
\]

Wir haben für die einzelnen Stäbe folgende Parameter bestimmt:


\begin{tabular}{rlll}
 \toprule
 & Stab 1 & Stab 2 & Stab 3 \\
 \midrule
Fitparameter:\\
 p0 &$0.031 \pm 0.002$ & $0.046 \pm 0.03$ & $0.0373 \pm 0.0023$\\ %TODO: Einheit?
 p1 & 0 (fest) &$ 0.16 \pm 1.7$ & $1.17 \pm 0.08$\\
 p2 & $0.65 \pm 0.19$ & & $-1.31 \pm 0.15$\\
Meßgrößen:\\
b (in cm) &$1,05 \pm 0,05$ & $1,05 \pm 0,05$ & $1,04 \pm 0,05$ \\
c (in cm) & $0,52 \pm 0,05$ & $0,52 \pm 0,05$ & $0,51 \pm 0,05$ \\
Elastizitätsmodul (in $10^9 \nicefrac{N}{m^2}$:\\
experimentell: & $82.2 \pm 10.7$ & $55,7 \pm 35,6$ & $(68.5 \pm 8.9) $ \\
theoretisch: & 195 & 100 & 72 \\
\bottomrule
\end{tabular}

Unsere Ergebnisse weichen teils gewaltig von den theoretischen ab, dies liegt vermutlich einerseits an perspektivischen Verzerrungen, andererseits ist die Biegung auch recht schwach, so dass alle Fits sehr linear aussahen und der kleine Anteil der Biegung unerheblich und willkürlich wirkte.

\section{Echtzeitholographie}
\subsection{Versuchsbeschreibung}
Bei der Echtzeitholographie wird ein Hologramm aufgenommen und dieses mit dem Bild des Objektes überlagert. Unterschiede zwischen dem Objekt und seinem Hologramm können dann als Interferenzmuster beobachtet werden, solange sie nicht zu groß sind. In unserem Experiment wollen wir mit dieser Methode die resonanten Schwingungen einer fest eingespannten Aluminiumplatte untersuchen. Dazu nehmen wir zuerst ein Hologramm von dieser auf und entwickeln dieses. Mit einem Lautsprecher und einem Frequenzgenerator wird dann die Platte in Schwingung versetzt. Das Interferenzmuster erlaubt dann Rückschlüsse auf die Verformung und zeigt bei den Resonanzfrequenzen ein charakteristisches Muster.

\subsection{Durchführung und Auswertung}

Wir haben den Aufbau analog wie im vorherigen Versuchsteil verwendet. Anstelle der Stäbe haben wir allerdings die Aluminiumplatte eingebaut und den Lautsprecher hinter dieser plaziert. Da das Interferenzmuster Änderungen im Wellenlängenbereich anzeigt, sollte die Photoplatte nicht bewegt werden. Die Aufnahme und Entwicklung findet daher "`in-situ"' statt, d.h. in einem Tauchbecken am Ort der Aufnahme. Dazu werden anstelle der Planfilme nun Photoplatten, also Glasscheiben mit einseitig aufgetragener Photoemulsion, verwendet. Für deren Prozessierung haben wir wiederum mündlich überlieferte \cite{lena_christian} Prozessparameter verwendet:

\begin{center}
\begin{table}[H]
\centering
 \begin{tabular}{l}
  \toprule
  10 min quellen\\
  6 min belichten\\
  4 min entwickeln\\
  10 sec vorwässer\\
  2 min wässern\\
  1 min bleichen\\
  3 x 30 sec wässern\\
  1 min wässern\\
  volllaufen lassen\\
 \bottomrule
 \end{tabular}
 \caption{Prozessschritte für die Erstellung von Holographien auf Photoplatten}
\end{table}
\end{center}

Unsere erste Aufnahme ist eher als Probelauf zu verstehen, da wir das Prinzip der Fixierung der Glasplatten mit Metallklammern nicht verstanden hatten. Ursache dafür war vermutlich das sich diese nicht beim Tauchbecken befanden, sondern erst von uns gesucht werden mussten. Die resultierende Aufnahme zeigte dann nich einmal das ungestörte Hologramm (nur Referenzbeleuchtung). Die zweite Aufnahme, diesmal mit Befestigung, schien ebenfalls kein Hologramm ergeben zu haben. Als wir jedoch am nächsten Tag wiederkamen und den Laser einschalteten, konnten wir klar und deutlich ein Hologramm der Aluminiumplatte erkennen. Auch hatte die Flüssigkeit im Tauchbecken in der Zwischenzeit eine braune Färbung angenommen. Es ist also vermutlich Entwickler durch den leicht undichten Hahn nachgelaufen und hat für eine Nachentwicklung gesorgt. Das Hologramm ist in Abbildung \ref{hologramm_aluplatte} zu sehen. Wir haben dann versucht, Schwingungsresonanzen der Platte aufzunehmen, konnten diese aber nicht finden. Vermutlich waren Tauchbecken und Aluminiumplatte in der Zwischenzeit durch Erschütterungen oder versehentliches Anstoßen schon gegeneinander verrückt. Dafür spricht auch, das wir die sonst bei ausgeschaltetem Lautsprecher sichtbaren Interferenzmuster auch nicht sehen konnten.  

\begin{figure}[H]
 \includegraphics[width=\textwidth]{Photos/IMG_3927.jpg}
 \caption{Hologramm der Aluminiumplatte}
 \label{hologramm_aluplatte}
\end{figure}

Nach dieser Aufnahme war leider die Bleiche nahezu aufgebraucht. Auch Herr Stützler konnte uns keinen Nachschub liefern, da die Chemikalienausgabe in dieser Woche urlaubsbedingt vakant war. Seinen Vorschlag die Reste aus dem Auffangbehälter zu recyceln mussten wir leider ebenfalls verwerfen, da Geruchs- und Sichtproben uns deutliche Hinweise auf eine Verunreinigung mit Entwicklerlösung gaben. Wir haben daher die verbleibenden Reste im Zulauf im Verhältnis 1:7 mit Wasser gestreckt. Zusätzlich haben wir noch die (teilweise schon auskristallisierten) Reste aus dem Bleichebecken für die Filmentwicklung hinzugegeben. Eine verlängerte Bleichezeit von 5min gab uns bei der dritten Photoplatte zwar kein vollkommen gebleichtes Bild, kam dem aber sehr nahe. Mit diesem waren jetzt auch im ungestörten Zustand der Platte Interferenzmuster erkennbar, wie wir es erwartet hatten.

\begin{figure}[H]
 \includegraphics[width=\textwidth]{Photos/IMG_3932.png}
 \caption{Aufnahme der Echtzeitholografie ohne Anregung durch den Lautsprecher}
 \label{echtzeit_ohne_ton}
\end{figure}

\begin{figure}[H]
 \includegraphics[width=\textwidth]{Photos/IMG_3934.png}
 \caption{Nullte Schwingungsmode der Kupferplatte bei 481Hz}
 \label{echtzeit_481}
\end{figure}


Daher haben wir nun den Frequenzgenerator angeschaltet und versucht die Resonanzfrequenzen zu finden. Für die nullte Schwingungsmode ist uns dies auch gut gelungen. Bei einer Frequenz $f_{00} = 481 Hz$ konnten wir klar ein konzentrisches Kreismuster im Hologramm (Abb. \ref{echtzeit_481}) erkennen. Außerdem konnten wir dies auch akustisch anhand der resonanten Verstärkung bzw. der Schwebung bei geringfügiger Abweichung bestätigen. Das Muster war im Frequenzbereich von 478Hz bis 485Hz erkennbar. Die Lautstärke mussten wir dabei sehr gering wählen, an der Grenze der Hörschwelle, da die Minima sonst nur deutlich schwächer zu sehen wären. Da wir einen neuen Versuchstisch benutzten, auf dem die Pockelszelle noch nicht montiert und insbesondere nicht justiert war, konnten wir nur die Echtzeitmittelungstechnik verwenden. Die Intensität folgt bei dieser der Funktion $|M(\delta)|^2 = 1 - J_0(\delta)$ (Abb. \ref{echtzeit_intensität}) wobei $\delta = \frac{2\pi}{\lambda}\bar{d}(\cos \alpha + \cos \beta)$ ein Maß für die Phasendifferenz und somit die Auslenkung ist. Wenn mit zunehmender Lautstärke also die Auslenkung ebenfalls zunimmt, so werden Abstand und Intensitätsänderung der Minima kleiner und das Bild ist schlechter zu erkennen. Beleuchtet man hingegen nur die Momente maximaler Auslenkung stroboskobisch mit dem durch die Pockelszelle gepulsten Laser, so könnte ein stehendes Bild erzeugen. Dieses wäre dann äquivalent zu einer Echtzeitholographie, bei der die Intensität der Funktion $|M(\delta)|^2 = \sin^2 \delta$ folgt. Man erhielte also komplette Minima, was den Kontrast deutlich verbessern würde.

\begin{figure}[H]
\includegraphics[width=\textwidth]{Graphen/echtzeitmittelung.pdf}
\caption{Normierte Intensität in Abhängigkeit von der Phasenverschiebung $\delta$ für Echtzeit- und Echtzeitmittelungsholografie}
\label{echtzeit_intensität}
\end{figure}

Die höheren Schwingungsordnungen konnten wir leider nicht auflösen. %TODO: ...
\section{Fourieroptik}
\subsection{Versuchsbeschreibung}

In diesem letzten Versuchsteil soll die Kreuzkorrelation und Faltung eines drehbaren Spaltes beobachtet werden. Dazu wird zunächst ein Hologramm der Fouriertransformierten dieses Spaltes erstellt. Nach der Entwicklung wird dieses dann ähnlich der Echtzeitspektroskopie mit der Fouriertransformierten des selben Spalts in verdrehter Einstellung überlagert. Je nach Betrachtungswinkel kann nun hinter der Platte entweder die die Faltung der beiden Apperturfunktionen (Spalte) oder deren Kreuzkorrelation betrachtet werden.

\begin{figure}[H]
 \includegraphics[width=\textwidth]{BilderAufbau/A-Bamberger-aufbau.png}
 \caption{Schematischer Versuchsaufbau für die Fourieroptik \cite{fourier}}
 \label{fourier_aufbau}
\end{figure}


\subsection{Durchführung und Auswertung}

Wir haben den Versuch wie in der Anleitung \cite{fourier} angegeben aufgebaut und ein Hologramm des Spaltes erstellt. Wir haben dabei die selben Parameter wie bei der Echtzeitholografie verwendet, die Belichtungszeit jedoch auf 3 min reduziert. Da bei diesem Aufbau mit dem direkten Strahl gearbeitet wird, ist die Intensität höher. Nach der Entwicklung waren sowohl das Hologramm des Spaltes als auch das Objektbild überlagert mit der Lupe zu sehen. Es war jedoch keine Wechselwirkung der beiden auszumachen, die zu einer Faltung oder Kreuzkorrelation geführt hätte.

\begin{figure}[H]
 \includegraphics[height=0.5\textheight]{Photos/IMG_3941.jpg}
 \caption{Versuchsaufbau für die Fourieroptik}
 \label{fourier_bild}
\end{figure}

Als Ursache für das Fehlen der Faltung und Kreuzkorrelation gibt es eine Vielzahl an Möglichkeiten. Dies fängt damit an, dass wir uns zum Beobachten auf das Fensterbrett legen mussten und sowohl Lupe als auch Augen freihändig in die richtige Position bringen mussten. Ein längerer Versuchstisch hätte hier vielleicht Abhilfe geschaffen, wir waren jedoch bereits froh überhaupt einen schwingungsgedämpften zur Verfügung zu haben. Wenn dieses Problem (zufällig) einmal überwunden ist, bereiten natürlich immernoch eventuelle Verrückungen am Aufbau während der Aufnahme Probleme. Weiterhin könnte es sein, das wir die Fourierebene nicht korrekt mit der Hologrammebene in Einklang gebracht haben.
Da wir auch die Winkel zwischen Objektstrahl und Referenzstrahl sowie die Position der Platte ungünstig gewählt hatten, haben wir den Aufbau entsprechend angepasst (Abb \ref{fourier_bild}). Leider haben wir nicht ausreichend Entwicklerflüssigkeit in das Becken laufen lassen, so das nur die Hälfte des Bildes entwickelt wurde. Natürlich viel uns dies erst nach dem Bleichen auf. Somit war jetzt auch die verdünnte Bleiche auf eine minimale Restmenge zusammengeschrumpft und eine weitere Streckung mit Wasser (dann ca. 50:1) erschien uns fragwürdig. Eine weitergehende Optimierung der Parameter wie Spaltöffnung, Winkel, Strahlengang, Belichtungszeit etc. war uns deshalb leider nicht mehr möglich.
%\section{Versuchsbeschreibung}

\begin{figure}
 \includegraphics[width=\textwidth]{BilderAufbau/spektrum.pdf}
 \caption{Versuchsaufbau für die Messung des Spektrums und das Einstellen der Fenster}
\end{figure}

\begin{figure}
 \includegraphics[width=\textwidth]{BilderAufbau/2er-koinzidenz.pdf}
 \caption{Versuchsaufbau für die Koinzidenzmessung des 2-Photonen-Zerfalls}
\end{figure}

\begin{figure}
 \includegraphics[width=\textwidth]{BilderAufbau/3er-koinzidenz.pdf}
 \caption{Versuchsaufbau für die Koinzidenzmessung des 3-Photonen-Zerfalls}
\end{figure}

%\section{Procedure and Analysis}

\subsection{Initiation and characteristics of the laser diode}

After switching on the laser diode and the thermoelectric cooler, we controlled the progression and the coherence of the laser beam and verified the correct position of the lenses, so that the beam focused on the detector of the photodiode. We did this by using an infrared display unit and a white piece of paper in order to be able to see the actual progression of the beam.\\

We then turned on the photodiode and the preamplifier, which we set to DC and the gain to 40 dB. We installed the etalon into the optical path and set it perpendicular to the beam. The laser was modulated by a sawtooth voltage in order to be able to see a few wavelength-peaks of the etalon.\\

In the first part of this measurement, the current of the diode was held at a constant value of $I = (35.1\pm 0.3)\ mA$ while the temperature was varied from $34.0^\circ C$ to $36.5^\circ C$. We measured the scrolling of the peaks in dependence of the temperature.

$$ AUSWERTUNG $$

In the second part of the measurement, we held the temperature constant at a value of $T=(34.7 \pm 0.2)\ ^\circ C$ and changed the current from 20.4 to 36.7 mA.

$$ AUSWERTUNG $$

\subsection{Hyperfine Structure}

We installed the Rubidium cell into the optical path and removed the etalon. The laser diode was modulated by a triangular voltage with a peak-to-peak difference of $\sim 200\ mV$, and the temperature was set to $T = (34.6 \pm 0.2)^\circ C$, as described in the instructions. However, with these adjustments we weren't able to find the hyperfine structure of the Rubidium atoms. Furthermore, the current of the laser diode wasn't able to surpass 35 mA and the preamplifier of the photodiode  was overcharged. We then tried different adjustments and installed a neutral filter (\emph{D 4,3}) into the optical path. The new adjustment was:\\

\begin{center}
\begin{tabular}[H]{l c}
Temperature & $T=34.4 ^\circ C$\\
Peak-to-Peak Voltage & $U_{pp} = 134\ mV$\\
\end{tabular}
\end{center}

We were then able to see the hyperfine structure at a current of $I = (62.9 \pm 0.3)\ mA$. With the same adjustments, we did another calibration. For this, however, we had to remove the filter, because the peaks weren't visible else.
\section{Zusammenfassung}

Wir konnten das \Na-Spektrum mit allen Szintillatoren aufnehmen und anhand der charakteristischen 511keV- und 1275keV-Linien eine Energie-Channel-Eichung vornehmen. Damit konnten wir das Maximum des Zwei-Photonen-Zerfalls auf $(178,3 \pm 0,01)^\circ$ bestimmen und somit die aus der Impulserhaltung folgende Annahme des Maximums bei $180^\circ$ bestätigen. Beim Drei-Photonen-Zerfall in $120^\circ$-Konfiguration haben wir ein Maximum der Ereignisse bei $(340 \pm 10)$keV festgestellt. Dies entspricht 
\section{Appendix}

\begin{appendix}
\section{Sources}

\begin{itemize}
\item Baur, C. : \emph{Einrichtung des Versuches ''Optisches Pumpen mit Laserdioden''}, Freiburg im Breisgau 1997 [Ba97]
\item Moseler, M., Pfaff, O. : \emph{Optisches Pumpen}, Freiburg im Breisgau 1990
\item Demtröder, W.: \emph{Experimentalphysik 3: Atome, Moleküle und Festkörper}, Kaiserslautern 2005
\end{itemize}

\end{appendix}

\clearpage

\section{Protocol}

\clearpage

\bibliographystyle{alphadin} 
\bibliography{holographie}


\end{document}
