\documentclass{beamer}

\usepackage[utf8]{inputenc} \usepackage[T1]{fontenc} \usepackage[ngerman]{babel} \usepackage[nice]{nicefrac} 
\usepackage{amsmath, amsfonts, amssymb, mathptmx, microtype, booktabs, graphicx, wrapfig, float, color, ulem, multimedia}
%\usepackage[unicode]{hyperref}

\usetheme{Warsaw}
\title[Optisches Pumpen]{OPTISCHES PUMPEN \\ Hyperfeinstrukturmessung mit dem Kompass}
\author{Robi Pedersen und Simon Schmeißer}
\institute{Albert-Ludwig-Universität Freiburg}
\date{24. (?) März 2010}

\begin{document}

\begin{frame}
\titlepage
\end{frame}

\begin{frame}[shrink]{Inhalt}
\tableofcontents
\end{frame}

%------------------------------------------------------------------------------------------------------------------------------------------------

\section{Aufgaben}

\begin{frame}{Aufgabenstellung}
\begin{itemize}
\item Hyperfeinstrukturspektrum des $^2S_{1/2} - ^2P_{1/2}$-Übergangs von $^{85}Rb$ und $^{87}Rb$.
\item Kernspin der Rubidium-Isotope anhand Doppelresonanz.
\item Bestimmen der Horizontal- und Vertikalkomponente des Erdmagnetfelds. 
\item Spinpräzession von ausgerichteten Atomen um das Erdmagnetfeld.
\item Relaxationszeit nach Dehmelt und nach Franzen. 
\end{itemize}
\end{frame}


\begin{frame}{Wandern der Maxima}
\begin{columns}
\begin{column}{5cm}
	\begin{figure}[H]
	\movie[poster, height = 3.5 cm, width = 4.8 cm]{Wandern der Maxima}{Movies/I-aendert.avi}
	\caption{Veränderung der Intensität}
	\end{figure}
\end{column}
\begin{column}{5cm}
	\begin{figure}[H]
	\movie[poster, height = 3.5 cm, width = 4.8 cm]{Wandern der Maxima}{Movies/T-aendert.avi}
	\caption{Veränderung der Temperatur}
	\end{figure}
\end{column}
\end{columns}
\end{frame}







\end{document}
