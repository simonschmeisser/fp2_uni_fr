\section{Aufgabenstellung}

\begin{enumerate}
\item Nehmen Sie ein direktes $^{22}Na$-Spektrum mit der PCA-Karte auf (SZ1). 
\item Untersuchen Sie die Richtungskorrelation des 2-$\gamma$-Zerfalls von Para-Positronium, indem Sie die Koinzidenzzählrate der beiden 511 keV Gammaquanten zwischen $\theta = 150^{\circ}$ und $180^{\circ}$  (beide Richtungen) betrachten (B = 0, Fenster zweier Szintillatoren auf 511 keV Linie setzen). 
\item Überlegen Sie sich die Anzahl der zufälligen Koinzidenzen aus $N_{zuf} = 2\cdot N_1\cdot N_2 \cdot \tau$ wobei $N_1$ , $N_2$ Zählraten in den Fenstern und $\tau$ die Auflösungszeit der Koinzidenz-Stufen bedeuten. Ermitteln Sie zusätzlich explizit die zufälligen Koinzidenzen (bei B = 0 T) durch Laufzeitverzögerung, indem Sie die untere und die obere Schwelle des SZ1 grob verzögern (1 $\mu$s oder mehr). 
\item Untersuchen Sie die Energieverteilung der Gammaquanten bei 3-fach-Koinzidenz in einer symmetrischen $120^{\circ}-120^{\circ}-120^{\circ}$-Geometrie, indem Sie bei den Szintillatoren SZ2 und SZ3 knapp unterhalb 
von 511 keV die obere Schwelle setzen und das Spektrum des Szintillators SZ1 uneingeschränkt betrachten. Überzeugen Sie sich davon, dass der 3-$\gamma$ -Zerfall des Ortho-Ps in alle Richtungen innerhalb einer Ebene stattfinden kann: Welche Energien tragen die drei $\gamma$-Quanten? Überlegen Sie sich, warum die $120^{\circ}-120^{\circ}-120^{\circ}$-Geometrie besonders günstig für die 3-fach-Koinzidenzmessung 
ist.
\item Ordnen Sie SZ1, SZ2 und SZ3 in der $120^{\circ}-120^{\circ}-120^{\circ}$-Geometrie an. Setzen Sie die Fenster aller drei Szintillatoren auf die 340 keV Linie. Zu beachten ist, dass die Impulshöhen bei starkem Magnetfeld etwas verschoben werden können. Registrieren Sie nun die 3-fach-Konzidenzzählrate mit und ohne Magnetfeld B. Führen Sie Messreihen mit Messzeiten t $\leq$ 2 h durch. Nach jeweils 1 h Messung mit Magnetfeld ist zur Prüfung der Stabilität eine Messung ohne Magnetfeld vorzunehmen. Zur Überwachung der unteren und oberen Schwellen von z.B. SZ2 und SZ3 dienen die beiden HEX-SCALER, deren Gates mit dem Ausgang des Koinzidenzzählers verbunden werden. 
\end{enumerate}

\clearpage