\section{Theorie}

\subsection{Bildung des Positroniums ($e^+ e^-$)}

Werden Positronen \positron in Materie gestoppt, so geben sie zunächst durch zahlreiche inelastische Stöße ihre Energie ab, bevor es mit Elektronen \elektron der äußeren Schalen zur Paarvernichtung unter Emission elektromagnetischer Strahlung kommt.
\begin{equation*}
 \positron + \elektron \rightarrow 2\gamma , 3\gamma
\end{equation*}

Dieser Prozess kann zur Abstrahlung von zwei oder drei Photonen führen, wobei der $e\gamma$-Zerfall ungefähr um den Faktor $\alpha = \frac{e^2}{\hbar c} \sim \frac{1}{137}$ unterdrückt ist. Der Wirkungsquerschnitt $\sigma_{2\gamma}$ eines ruhenden Elektrons, in nezug auf Vernichtung mit dem Positron in 2 Quanten, lautet für nichtrelativistische Geschwindigkeiten ($ v << c$) [LIT: DIRAC 30]:
\begin{equation*}
 \sigma_{2\gamma} = \frac{\pi r_0^2 c}{v}
\end{equation*}
mit
\begin{equation*}
 r_0 = \text{Thomson-Radius} = \frac{e^2}{mc^2} = 2,8\e{-13}\text{cm}
\end{equation*}

\begin{equation*}
 v = \text{Relativgeschwindigkeit von \positron und \elektron}
\end{equation*}

Dann ist die Vernichtungswahrscheinlichkeit eines Positrons in einem Medium, daß n Elektronen pro $\text{cm}^3$ enthält, gegeben durch:
\begin{equation*}
 \lambda = \frac{1}{\tau} = \sigma_{2\gamma} \cdot n \cdot v = \pi r_0^2 c n   \tau = \text{mittlere Lebensdauer}
\end{equation*}

woebi die Coulomb-Anziehung zwischen den Teilchen vernachlässigt wird.

Dieser direkten Vernichtung eines Positron-Elektron-Paares kann bei noedrigen Geschwindigkeiten die Entstehung des kurzlebigen Positronium-Atoms (\positron \elektron) vorrausgehen:

\begin{equation*}
 \positron + \elektron \rightarrow \positron \elektron \rightarrow 2\gamma, 3\gamma
\end{equation*}

Es sind nur Positronen eines schmalen Energiebereichs in der Lage, sich ein Elektron einzufangen und Positronium im Grundzustand zu bilden.
Da die Ionisierungsenergie von $6,8 \text{eV}$ des Positroniums kleiner als diejenige aller stabilen Gase ist, können Positronen deren $E_\text{kin} < (V_i - 6,8 \text{eV}$ ($V_i$ = Ionisationspotential der Gasmoleküle) ist, kein Positronium bilden. Andererseits sind aber bei großen kinetischen Energien die Wirkungsquerschnitte für Stoßionisation und Anregung der Gasmoleküle größer als bei den Einfangprozessen zur Bildung von Positronium. Eine Entstehung im 1. angeregten Zustand ist unter gewöhnlichen experimentellen Bedingungen sehr unwahrscheinlich, da die Bindungsenergie nur $1,7 \text{eV}$ beträgt. Das bedeutet, daß das Minimum der zum Einfangen eines Elektrons erforderlichen kinetischen Energie ($V_i - 1,7 \text{eV}$) bereits so groß ist, daß inelastische Stöße stark überwiegen. Wurde ien Positronium gebildet, do kann es natürlich in molekularen Stößen wieder in ein Positron und ein Elektron auseinanderbrechen. Da die mittlere freie Weglänge in Festkörpern kleiner ist als in verdünnten Medien, kann man in den amorphen Substanzen Teflon und geschmolzener Quarz nur ungefähr 1 \% Positronium nachweisen. Deshalb ist es vorteilhafter mit speziellen Gasen, wie z.B. Schwefelhexafluorid $\text{SF}_6$ oder Freon $\text{C Cl}_2\text{F}_2$, zu arbeiten, bei denen man einen Anteil von ungefähr 30 \% erhält.

\subsection{Vernichtungscharakteristik des Positroniums}

Der Grundzustand des Positroniums 1S (l=0) wird je nach Spineinstellung in 2 Arten unterschieden:
\begin{enumerate}
 \item[a)] - Spins von Elektronen \elektron und Positronen \positron parallel
Gesamtspin $S = S_\elektron + S_\positron = 1$
Bezeichnung: Triplett-Positronium oder Orthopositronium (${}^3S_1$, $m_S = 0$, $m_S = \pm 1$)
$m_S$ = magnetische Spinquantenzahl
Statistisches Gewicht: $\frac{3}{4}$

 \item[b)] - Spins von \elektron und \positron antiparallel
S= 0
Bezeichnung: Singulett-Positronium oder Parapositronium (${}^1S_0$, $m_S = 0$)
Statistisches Gewicht: $\frac{1}{4}$ 
\end{enumerate}

Um im Vernichtungsprozeß den Erhaltungssätzen zu genügen, unterscheiden sich Triplett- und Singulett-Zustände erheblich in ihrer Vernichtungscharakteristik
\begin{equation*}
 \positron + \elektron \rightarrow 2\gamma , 3\gamma
\end{equation*}
Wegen der verletzung der Energie- und Impulserhaltung ist ein $1\gamma$-Zerfall im freien Raum nicht möglich.
Der $1^1S$-Zustand hat eine mittlere Lebensdauer $\tau_{2\gamma}$ von:
\begin{equation*}
 \frac{1}{\tau_{2\gamma}} = \lambda = \sigma_{2\gamma} \cdot v \cdot \left| \Psi(0) \right|^2 = \frac{\alpha^5 m c^2}{2 \hbar}
\end{equation*}
\begin{equation*}
 \text{mit} \left| \Psi(0) \right|^2 = \frac{1}{\pi} \left( \frac{1}{2a_0} \right)^3
\end{equation*}
Dichte der 1S-Elektronen am Positron, Bohrscher Radius $a_0 = \frac{r_0}{\alpha^2}$
Numerisch: $\tau_{2\gamma} = 1,25\e{-10}\text{sec}$

Unter Berücksichtigung der Erhaltungssätze zerstrahlt er in zwei Photononen, die sich in entgegengesetzter Richtung mit je 0,511 MeV ausbreiten und entweder im gleichen Sinne zirkularpolarisiert oder zueinander orthogonal linearpolarisiert sind.

Für den ${}^3S$-Zustand verbieten mit der Symmetrie des Systems verknüpfte Auswahlregeln den Zerfall in zwei Photononen. Der wahrscheinlichste Zerfallsvorgang des Orthopositroniums ist ein Strahlungsprozeß dritter Ordnung. Es zerstrahlt in drei Quanten, die wegen der Impulserhaltung in einer Ebene emittiert werden und jeweils ein kontinuirliches Energiespektrum von 0 bis 2 $m_ec^2$ besitzen, so daß die Gesamtenergie $E_{3\gamma} = 1,022 MeV$ gleich der gesamten Ruheenergie der beiden einander vernichtenden Teilchen entspricht. Weiterhin ist die Dreiquantenvernichtung gekennzeichnet durch eine kleinere Zerfallsrate $\lambda$ als die Zweiquantenvernichtung. Für das Verhältnis dieser Raten sind mehrere Berechnungen gemacht worden [LIT:Ore 49]; es ergibt sich zu:
\begin{equation*}
 \frac{\lambda_{3\gamma}}{\lambda_{2\gamma}} = \frac{\tau_{2\gamma}}{\tau_{3\gamma}} = \frac{4}{9\pi} \left(\pi^2 - 9 \right) \alpha = \frac{1}{1115}
\end{equation*}

Die mittlere Lebensdauer der $1^3S$-Zustands ist:
\begin{equation*}
 \tau_{3\gamma} = 1,39\e{-7} \text{sec}
\end{equation*}

Die Entstehung der Vernichtungsauswahlregeln, die sich aus den Invariantenbedingungen der Operation Ladungskonjugation C und Parität P ergeben, wird im nächsten Abschnitt näher behandelt.

\subsubsection{Symmetrieauswahlregeln}
Zunächst soll gezeigt werden, daß der Singulett-Zustand des Positroniums noicht durch Emission von drei Photonen oder jeder anderen ungeraden Anzahl von Quanten vernichtet werden kann. Die Operation der Ladungskonjugation C transformiert das Teilchen in sein Antiteilchen und kehrt das Vorzeichen der elektromagnetischen Felder um.
Für das nach außenhin elektrisch neutrale System des Positroniums hat die Ladungskonjugation C einen definierten Eigenwert $C=\pm1$, der beim Positronium-Zerfall erhalten bleiben muß. Für das Photon gilt ungerade C-Parität $C=-1$ , und da C eine multiplikative Quantenzahl ist, besitzt das System der n Vernichtungsquanten den Eigenwert $(-1)^n$.

Die C-Parität des Anfangszustands gewinnt man aus der folgenden Überlegung. Es kann nämlich gezeigt werden, daß die Operation C, angewandt auf das Positron-Elektron-Paar, äquivalent dem Ladungsaustausch ist, der seinerseits Parität P x Spinaustausch $S_{ex}$ entspricht:

$C | \positron \elektron >$ = Teilchenaustausch =(TODO:entspricht) Ladungsaustausch = P x Spinaustausch $S_{ex}$

Zum Beweis dieser Äquivalenz sei folgender Zustand betrachtet (s.Skizze 1):
\begin{itemize}
 \item[-] \elektron befindet sich am Ort $\vec{r}$ mit Spin $\alpha$
 \item[-] \positron befindet sich am Ort $-\vec{r}$ mit Spin $\beta$
\end{itemize}

1. Operation: Anwendung des Paritätsoperators P
 (= TODO:entspricht Inversion aller Raumkoordinaten)

Ergebnis:
\begin{itemize}
 \item[-] \elektron befindet sich am Ort $-\vec{r}$ mit Spin $\alpha$
 \item[-] \positron befindet sich am Ort $\vec{r}$ mit Spin $\beta$
\end{itemize}
Da die Spins axiale Vektoren sind, werden sie bei räumlicher Spiegelung nicht verändert.

2. Operation: Spin-Austausch
Ergebnis:
\begin{itemize}
 \item[-] \elektron befindet sich am Ort $-\vec{r}$ mit Spin $\beta$
 \item[-] \positron befindet sich am Ort $\vec{r}$ mit Spin $\alpha$
\end{itemize}
Offensichtlich erhält man nach beiden Operationen einen Zustand, in dem lediglich die Ladungen gegenüber dem Anfangszustand vertauscht sind.

TODO: SEITE 9 EINFÜGEN

Jetzt werden die beiden Operationen, die der Ladungskonjugation C äquivalent sind, auf den Positronium-Zustand angewendet.

\begin{itemize}
 \item[a)] Paritätsoperator P:
Der Ortsanteil der Wellenfunktion, ausgedrückt in Kugelkoordinaten, hat $(-1)^l$-Symmetrie (l = Bahnquantenzahl).
Die Gesamtparität ist:
\begin{equation*}
 P = \eta_i( \positron ) \cdot \eta_i (\elektron) \cdot (-1)^l
\end{equation*}
$\eta_i$ = definierte Eigenparität der einzelnen Elementarteilchen
\begin{equation*}
 \eta_i ( \elektron ) = - \eta_i ( \positron )
\end{equation*}

Beim Positron im Grundzustand (l = 0) erhalten wir also bei Inversion der Raumkoordinaten den Faktor:
\begin{equation*}
 P = (-1) \cdot (+1) \cdot (+1)
\end{equation*}

\item[b)] Spinaustausch $S_{ex}$:
Beim Austausch der Spinkoordinaten von Elektron und Positron tritt in der Spinwellenfunktion ein Faktor $(-1)^{s+l}$ auf (S = Gesamtspin). Für den Singulett-Zustand (Spins antiparallel, S = 0) bedeutet das, daß die Spinwellenfunktion antisymmertrisch ist.
\begin{equation*}
 S_{ex} = (-1)
\end{equation*}
\end{itemize}

Insgesamt besitzt der Singulett-Zustand folgende C-Parität:
\begin{equation*}
 C = P \cdot S_{ex} \\
 C = (-1)(+1)(+1)\cdot(-1) \\
 C = (+1)
\end{equation*} %TODO: Mehrzeilige Gleichung

Da der Ladungskonjugations-Operator C eine Konstante der Bewegung ist $ \frac{dC}{dt} = [H, C] = 0$, kann der Singulett-Zustand also nur in zwei Quanten zerstrahlen; denn der C-Eigenwert für die Photonen beträgt $C = (-1)^n$. Ganz analog erfolgt die Berechnung für den Triplett-Zustand, aus welcher hervorgeht, daß er nur in eine ungerade zahl von Quanten zerfallen kann.

\subsection{Die Energieniveaus des Positroniums}

\subsubsection{Grobstruktur}

Positronium ist ein wasserstoffähnliches Atom, in dem das Positron den platz des Protons eingenommen hat. Die stationären Zustände dieses stabilen Systems werden auf die gleiche Weise berechnet wie die für das H-Atom.
Bei Benutzung des nichtrelativistischen Hamilton-Operators
\begin{equation*}
 H_0 = \frac{P_{\elektron}^2}{2m} + \frac{P_{\positron}^2}{2m} - \frac{e^2}{r}
\end{equation*}
Masse: $m_\positron = m_elektron = m$
P = Impulsoperator

erhält man aus der Schrödingergleichung folgende Energiezustände:
\begin{equation*}
 E_n = - \frac{m_e^4}{4 \hbar^2 n^2} = - \frac{Ry}{2 n^2}
\end{equation*}
mit Rydbergkonstante für Positronium $R_{y_p}$: $R_{y_p} = \frac{1}{2} Ry = 6,8 eV$

Im Vergleich zum H-Atom zeigt sich, daß wegen der reduzierten Masse des Positroniums $\mu_r = \frac{m_\positron \cdot m_\elektron}{m_\positron + m_\elektron} = \frac{m}{2}$, die Rydberg-Konstante um den Faktor $\frac{1}{2}$ kleiner ist. Dementsprechend müssen alle Energieniveaus des H-Atoms mit $\frac{1}{2}$ multipliziert werden, um z.B. die Bindungsenergie des Positronium-Grundzustandes von 6,8 eV und die des 1. angeregten Zustands von 1,7 eV zu erhalten.

\subsubsection{Feinstruktur des Positroniums}

Das Interesse gilt jetzt der Feinstruktur der Energieniveaus, die allgemein in Atomsystemen durch die Kopplung des Elektronspins $\vec{S}$ mit dem Bahndrehimpuls $\vec{L}$ zum Gesamtdrehimpuls $\vec{J} = \vec{L} + \vec{S}$ hervorgerufen wird. Eine Hyperfeinstruktur (HFS), die durch die Wechselwirkung der magnetischen Spinmomente entsteht, ist gewöhnlich um eine Ordnung kleiner. Jedoch in Positronium sind die Hyperfeinstrukturterme von derselben Größenordnung wie die Feinstrukturterme in Wasserstoff, da (Weiter seite 12)