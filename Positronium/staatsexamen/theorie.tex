\section{Theorie}

\subsection{Bildung des Positroniums ($e^+ e^-$)}

Werden Positronen \positron in Materie gestoppt, so geben sie zunächst durch zahlreiche inelastische Stöße ihre Energie ab, bevor es mit Elektronen \elektron der äußeren Schalen zur Paarvernichtung unter Emission elektromagnetischer Strahlung kommt.
\begin{equation*}
 \positron + \elektron \rightarrow 2\gamma , 3\gamma
\end{equation*}

Dieser Prozess kann zur Abstrahlung von zwei oder drei Photonen führen, wobei der $e\gamma$-Zerfall ungefähr um den Faktor $\alpha = \frac{e^2}{\hbar c} \sim \frac{1}{137}$ unterdrückt ist. Der Wirkungsquerschnitt $\sigma_{2\gamma}$ eines ruhenden Elektrons, in nezug auf Vernichtung mit dem Positron in 2 Quanten, lautet für nichtrelativistische Geschwindigkeiten ($ v << c$) [LIT: DIRAC 30]:
\begin{equation*}
 \sigma_{2\gamma} = \frac{\pi r_0^2 c}{v}
\end{equation*}
mit
\begin{equation*}
 r_0 = \text{Thomson-Radius} = \frac{e^2}{mc^2} = 2,8\e{-13}\text{cm}
\end{equation*}

\begin{equation*}
 v = \text{Relativgeschwindigkeit von \positron und \elektron}
\end{equation*}

Dann ist die Vernichtungswahrscheinlichkeit eines Positrons in einem Medium, daß n Elektronen pro $\text{cm}^3$ enthält, gegeben durch:
\begin{equation*}
 \lambda = \frac{1}{\tau} = \sigma_{2\gamma} \cdot n \cdot v = \pi r_0^2 c n   \tau = \text{mittlere Lebensdauer}
\end{equation*}

woebi die Coulomb-Anziehung zwischen den Teilchen vernachlässigt wird.

Dieser direkten Vernichtung eines Positron-Elektron-Paares kann bei noedrigen Geschwindigkeiten die Entstehung des kurzlebigen Positronium-Atoms (\positron \elektron) vorrausgehen:

\begin{equation*}
 \positron + \elektron \rightarrow \positron \elektron \rightarrow 2\gamma, 3\gamma
\end{equation*}

Es sind nur Positronen eines schmalen Energiebereichs in der Lage, sich ein Elektron einzufangen und Positronium im Grundzustand zu bilden.
Da die Ionisierungsenergie von $6,8 \text{eV}$ des Positroniums kleiner als diejenige aller stabilen Gase ist, können Positronen deren $E_\text{kin} < (V_i - 6,8 \text{eV}$ ($V_i$ = Ionisationspotential der Gasmoleküle) ist, kein Positronium bilden. Andererseits sind aber bei großen kinetischen Energien die Wirkungsquerschnitte für Stoßionisation und Anregung der Gasmoleküle größer als bei den Einfangprozessen zur Bildung von Positronium. Eine Entstehung im 1. angeregten Zustand ist unter gewöhnlichen experimentellen Bedingungen sehr unwahrscheinlich, da die Bindungsenergie nur $1,7 \text{eV}$ beträgt. Das bedeutet, daß das Minimum der zum Einfangen eines Elektrons erforderlichen kinetischen Energie ($V_i - 1,7 \text{eV}$) bereits so groß ist, daß inelastische Stöße stark überwiegen. Wurde ien Positronium gebildet, do kann es natürlich in molekularen Stößen wieder in ein Positron und ein Elektron auseinanderbrechen. Da die mittlere freie Weglänge in Festkörpern kleiner ist als in verdünnten Medien, kann man in den amorphen Substanzen Teflon und geschmolzener Quarz nur ungefähr 1 \% Positronium nachweisen. Deshalb ist es vorteilhafter mit speziellen Gasen, wie z.B. Schwefelhexafluorid $\text{SF}_6$ oder Freon $\text{C Cl}_2\text{F}_2$, zu arbeiten, bei denen man einen Anteil von ungefähr 30 \% erhält.

\subsection{Vernichtungscharakteristik des Positroniums}

Der Grundzustand des Positroniums 1S (l=0) wird je nach Spineinstellung in 2 Arten unterschieden:
\begin{enumerate}
 \item[a)] Spins
 \item[b)] Spins 
\end{enumerate}
