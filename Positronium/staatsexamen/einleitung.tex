\section{Einleitung}

Positronium ist der gebundene Zustand des Elektrons mit seinem Antiteilchen, dem Positron. Allgemein hat das experimentelle Erforschen dieses 2-Teilchen-Systems zur Bestätigung vieler Vorhersagen der Quantenelektrodynamik geführt. Die Feinstruktur dieses Atoms wird nicht allein durch einen magnetischen Spin-Spin Wechselwirkungsterm bestimmt. Vielmehr forderten theoretische Analysen, daß zusätzlich eine größenordnungsmäßig gleiche Wechselwirkung, die "`Annihilationskraft"', existiertwelche die virtuelle Vernichtung berücksichtigt.

Mit Hilfe dieses Experiments soll die Feinstrukturaufspaltung $\Delta W$ des Positronium Grundzustands $1 S$ bestimmt werden. Dazu wird das "`Quenching"' (Löschen) der Dreiquantenvernichtung des Positroniums gemessen, indem man die Abnahme der Dreifachkoinzidenzrate im Magnetfeld beobachtet. 