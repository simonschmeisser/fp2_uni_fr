\section{Zusammenfassung}

Wir konnten das \Na-Spektrum mit allen Szintillatoren aufnehmen und anhand der charakteristischen 511keV- und 1275keV-Linien eine Energie-Channel-Eichung vornehmen. Damit konnten wir das Maximum des Zwei-Photonen-Zerfalls auf 

$$(178,3 \pm 0,01)^\circ$$

bestimmen und somit die aus der Impulserhaltung folgende Annahme des Maximums bei $180^\circ$ bestätigen.\\ 

Beim Drei-Photonen-Zerfall in $120^\circ$-Konfiguration haben wir ein Maximum der Ereignisse bei $(340 \pm 10)$keV festgestellt. Dies entspricht genau dem was wir aus der Theorie in dieser Anordnung erwarten, nämlich einem Drittel der Positroniummasse (340.67 keV, 1.Standardabweichung) und wir konnten somit den 3-Photonen-Zerfall des Positroniums bestätigen.\\

Schlussendlich haben wir den Zerfall des Positroniums in 3 Photonen in Abhängigkeit eines extern wirkenden Magnetfelds gemessen und konnten die Theorie des Quenchings bestätigen und auch dessen Verlauf wiedergeben. Aus dem Fit auf die Messwerte konnten wir dann die Aufspaltung

$$\Delta W = (9.068 \pm 0.66) \cdot 10^{-4} $$

messen, die uns zeigt, dass neben der Hyperfeinstrukturaufspaltung ein weiterer Effekt vorliegt, nämlich die Annihilationskraft. Der theoretische Wert der Hyperfeinstrukturaufspaltung und der Annihilationskraft $\Delta W =8.45\cdot10^{-4}$ liegt auch innerhalb der ersten Standardabweichung.